\documentclass[light]{lutbeamer} % change between light and dark for the background
%\documentclass[t]{lutbeamer} % use "t" option for top alignment 
\usepackage{caption}
\usepackage{xcolor}
\captionsetup{labelfont={color=gr},textfont={color=gr}}
\DeclareCaptionLabelFormat{nocolon}{#1 #2}
\captionsetup{labelformat=nocolon}
\usepackage{pgfpages}
\usepackage{amssymb}
\usepackage{amsmath}
\usepackage{tabularx}
\usepackage{array}
\usepackage{adjustbox}
\usepackage{hyperref} % Link
\usepackage{bm}
\usepackage{amsfonts}
\usepackage{algorithmic}
\usepackage{textcomp}
\usepackage{xcolor}
\usepackage{algorithm} 
\usepackage{amsthm} % add this package to use the definition environment

\setbeameroption{hide notes} % Only slides
% \setbeameroption{show only notes} % Only notes
% \setbeameroption{show notes on second screen=right} % Both


\setdepartment{Data Science and Analytics Thrust, Information Hub}
\institute[HKUSTGZ]{The Hong Kong University of Science and Technology (Guangzhou)}
\author{Mingze Gong}
\title{Spatio-Temporal Forecasting}
\subtitle{Deterministic Graph Neural Networks for Carbon Emissions and Generative Probabilistic Stochastic Differential Equation-based Diffusion for Traffic Flow}
\date{\today}


\begin{document}

% front page
{ % all template changes are local to this group.
\setbeamertemplate{navigation symbols}{}
\begin{frame}<article:0>[plain,noframenumbering]
    \begin{tikzpicture}[remember picture,overlay]
        \node[at=(current page.center)] {
            \includegraphics[
                width=\paperwidth,
                height=\paperheight]{figures/GZ Campus.jpeg}
        };
    \end{tikzpicture}
\end{frame}
}

% Outline
\AtBeginSection[]
{
    \begin{frame}[plain,noframenumbering]
        \frametitle{Outline}
        \begin{columns}[T]
            \begin{column}{0.01\textwidth}

            \end{column}
            \begin{column}{0.95\textwidth}
                \tableofcontents[currentsection,
                    %currentsubsection,
                    %hideothersubsections, 
                    %sectionstyle=show/sh ed, 
                    %subsectionstyle=show/shaded%/hide
                ]
            \end{column}
        \end{columns}
    \end{frame}
}

{ % title page
    \begin{frame}[plain]
        \maketitle
        \small
        \par\vskip-0.1em
        {\footnotesize
        \begin{beamercolorbox}[sep=8pt,left]{author}
            \usebeamerfont{author}{Presented by \insertauthor} on \insertdate
        \end{beamercolorbox}%\vskip0.5em
        }
        \note{
            Dear Professors, Dr. Li and Dr. Zhang, Good morning, given the group project, I will be presenting my individual topic. \\~\\

            I will start with the project overview.
        }

    \end{frame}
}
% % % % % % % % % % % % % % % % % % % % % % % % % % % % % % % % % % % %
\section{Introduction}
\subsection{Research Overview}
\begin{frame}
    \frametitle{Research Overview}
    \framesubtitle{Investigations on Deterministic and Probabilistic Forecasting}
    \begin{columns}
        \begin{column}{0.5\textwidth}
            \includegraphics
            [width=\textwidth]{figures/1day_us_0region_modified_new.png}
            \captionof{figure}{A glimpse of forecasts comparison for deterministic carbon emissions forecasting}
        \end{column}
        \begin{column}{0.5\textwidth}
            \includegraphics
            [width=\textwidth]{figures/node200_pems04.png}
            \captionof{figure}{A glimpse of forecasts comparison for probabilistic traffic flow forecasting}
        \end{column}
    \end{columns}
\end{frame}

% % % % % % % % % % % % % % % % % % % % % % % % % % % % % % % % % % % %

\section{Deterministic Carbon Emissions Modeling}
\subsection{Introduction to DSTGCRN}
\begin{frame}
    \frametitle{Introduction}
    \framesubtitle{DSTGCRN: Background and Motivation}
    \begin{columns}[T]
        \begin{column}{0.45\textwidth}
            \textbf{Background:}
            \begin{itemize}
                \item Increase in carbon emissions due to fossil fuels and land degradation.
                \item Accurate forecasting is vital to:
                      \begin{itemize}
                          \item Inform sustainable policies.
                          \item Meet reduction targets:
                                \begin{itemize}
                                    \item China: Peak by 2030.
                                    \item US: Reduce 50-52% by 2030.
                                    \item EU: Cut 55% by 2030.
                                \end{itemize}
                      \end{itemize}
            \end{itemize}
        \end{column}
        \begin{column}{0.45\textwidth}
            \textbf{Challenges:}
            \begin{itemize}
                \item Models often overlook non-linear trends and regional dependencies.
                \item Forecasting difficulties:
                      \begin{itemize}
                          \item Variability within regions impacts total emissions.
                          \item Dynamics between regions significantly influence local emissions.
                      \end{itemize}
            \end{itemize}
        \end{column}
    \end{columns}
\end{frame}


\begin{frame}
    \frametitle{Introduction}
    \framesubtitle{DSTGCRN: Motivations and Contributions}
    \begin{columns}[T] % Aligns the top of the content in both columns
        \begin{column}{0.45\textwidth}
            \textbf{Motivations:}
            \begin{itemize}
                \item Overcome limitations of traditional methods in handling complex, dynamic environmental data.
                \item Leverage insights from the success of Graph Neural Networks (GNNs) in sectors such as traffic and energy to enhance spatial-temporal analysis.
            \end{itemize}
        \end{column}
        \begin{column}{0.45\textwidth}
            \textbf{Contributions:}
            \begin{itemize}
                \item Combines Graph Convolutional Networks (GCN) and Recurrent Neural Networks (RNN) to model evolving inter-regional relationships and spatial-temporal dynamics.
                \item Boosts predictive accuracy and offers comprehensive insights to guide environmental policy.
                \item Facilitates informed, real-time policy decisions adapted to specific regional contexts.
            \end{itemize}
        \end{column}
    \end{columns}
\end{frame}

\subsection{Literature Review}

\begin{frame}
    \frametitle{Literature Review}
    \framesubtitle{Statistical and Machine Learning Approaches}
    \begin{itemize}
        \item \textbf{Statistical Methods:}
              \begin{itemize}
                  \item \textbf{ARIMA Models:} Employed for time series forecasting; adjusts for trends and seasonality.
                  \item \textbf{Grey Forecasting Models (GM):} Effective under conditions of limited or incomplete data, applicable in emerging markets.
                  \item \textbf{Hybrid Models:} Combining GM and ARIMA to address non-linear and non-stationary data, enhancing forecast accuracy.
              \end{itemize}
        \item \textbf{Machine Learning Methods:}
              \begin{itemize}
                  \item \textbf{Deep Learning:} Excels in learning complex data patterns, significantly improving prediction capabilities.
                  \item \textbf{Hybrid Approaches:} Integration of neural networks with statistical methods boosts accuracy and reliability.
                  \item \textbf{Regional Variability:} Challenges include accommodating diverse environmental conditions, impacting scalability and model performance.
              \end{itemize}
    \end{itemize}
\end{frame}

\begin{frame}
    \frametitle{Literature Review}
    \framesubtitle{Advancements in Spatial-Temporal Predictions}
    \begin{itemize}
        \item \textbf{Spatial-Temporal Graph Neural Networks (STGNNs):}
              \begin{itemize}
                  \item At the cutting edge for modeling dynamic interdependencies across locations and times, crucial for precise environmental forecasts.
                  \item Adaptive graph structures in these models allow responsiveness to temporal changes, enhancing long-term prediction reliability.
              \end{itemize}
        \item \textbf{Attention Mechanisms:}
              \begin{itemize}
                  \item Prioritize crucial features and time steps, improving focus on significant data and reducing irrelevant noise.
                  \item Demonstrates enhanced detail and accuracy in environmental data analysis, effectively managing spatial and temporal dimensions.
              \end{itemize}
    \end{itemize}
\end{frame}

\subsection{Methodology}

\begin{frame}
    \frametitle{Methodology}
    \framesubtitle{Problem Definitions and Setup}
    \begin{itemize}
        \item \textbf{Multisource Time Series Forecasting:}
              Forecast future values using data from multiple regions on features like temperature and AQI.
              \[
                  \text{Forecast } \bm{Y}_{t+1}, \dots, \bm{Y}_{t+Q} \text{ using } f:\mathbb{R}^{N \times P \times C} \rightarrow \mathbb{R}^{N \times Q}
              \]
              Here, \( \bm{Y}_t \in \mathbb{R}^N \) denotes the output vector at time \( t \), \( P \) is the number of past time steps considered, and \( C \) represents the number of features per step.

        \item \textbf{Regional Carbon Emission Network:}
              Models interdependencies among regions through a graph structure to enhance predictive accuracy by integrating spatial dynamics.
              \[
                  \mathcal{G} = (\mathcal{V}, \mathcal{E}, \bm{A}), \quad \bm{A}_{ij} =
                  \begin{cases}
                      1 & \text{if there is a direct connection between regions } i \text{ and } j, \\
                      0 & \text{otherwise.}
                  \end{cases}
              \]
              Here, \( \mathcal{V} \) are nodes (regions), \( \mathcal{E} \) are edges (connections), and \( \bm{A} \) is the adjacency matrix showing regional connections.
    \end{itemize}
\end{frame}



\begin{frame}
    \frametitle{Methodology}
    \framesubtitle{AGCRN-Core Modeling Approach}
    \begin{itemize}
        \item \textbf{Node Embedding and Graph Convolution:}
              \[
                  \bm{X}'_t = \left(\bm{I}_N + \text{softmax}\left(\text{ReLU}\left(\bm{E} \cdot \bm{E}^\top\right)\right)\right) \bm{X}_t \bm{\Theta}
              \]
              where \(\bm{I}_N\) is the identity matrix, \(\bm{\Theta}\) is the weight matrix.
        \item \textbf{Integration with GRU:}
              \begin{equation*}
                  \begin{split}
                      \tilde{\bm A} &= \text{softmax}(\text{ReLU}(\bm E\cdot \bm E^T)) \\
                      {\bm R_t} &= \sigma\left(\tilde{\bm A}[\bm X_t, \bm H_{t-1}]\bm E\bm W_r + \bm E\cdot\bm b_r \right)
                      \\
                      {\bm U_t} &= \sigma\left(\tilde{\bm A}[\bm X_t, \bm H_{t-1}]\bm E\bm W_u + \bm E\cdot\bm b_u \right)
                      \\
                      \hat{\bm H}_t &= \tanh\left(\tilde{\bm A}[\bm X_t, \bm U_t\odot\bm H_{t-1}]\bm E\bm W_h + \bm E\cdot\bm b_h \right)
                      \\
                      \bm H_t &= \bm R_t\odot \bm H_{t-1} + (1-\bm R_t)\odot \hat{\bm H}_t
                  \end{split}
              \end{equation*}
    \end{itemize}
\end{frame}


\begin{frame}
    \frametitle{Methodology}
    \framesubtitle{Dynamic Spatial-Temporal Modeling}
    \begin{itemize}
        \item \textbf{Dynamic Embeddings:}
              Generate node embeddings that evolve over time, reflecting changing regional interdependencies. The update mechanism for these embeddings is given by:
              \[
                  \bm{E}_t = \text{DynamicEmbedding}(\bm{\mathcal{X}}_t), \quad \bm{\mathcal{X}}_t \in \mathbb{R}^{P \times N \times C}
              \]
              where \( \bm{\mathcal{X}}_t \) represents the input features across \( P \) past time steps, \( N \) regions, and \( C \) features.

        \item \textbf{Multihead Attention:}
              Applies multihead attention to capture distinct temporal patterns, enhancing the model's predictive accuracy. The mechanism is defined as:
              \[
                  \text{Attention}(\mathbf{Q}, \mathbf{K}, \mathbf{V}) = \text{softmax}\left(\frac{(\bm{\mathcal{X}}''_t \bm{W}_Q)(\bm{\mathcal{X}}''_t \bm{W}_K)^\top}{\sqrt{d_e}}\right)\bm{\mathcal{X}}''_t \bm{W}_V
              \]
              where \(\mathbf{Q}, \mathbf{K}, \mathbf{V}\) are the queries, keys, and values, respectively, transformed by the weight matrices \(\bm{W}_Q, \bm{W}_K, \bm{W}_V\), \( \bm{\mathcal{X}}''_t \) is the processed input, and \( d_e \) denotes the embedding dimension.
    \end{itemize}
\end{frame}


\begin{frame}
    \frametitle{Methodology}
    \framesubtitle{Framework Visualization}
    \begin{figure}
        \centering
        \includegraphics[width=0.8\textwidth]{figures/dstgcrn_framework.pdf}
        \caption{The architecture of the Dynamic Spatial-Temporal Graph Convolutional Recurrent Network (DSTGCRN).}
    \end{figure}
\end{frame}

\subsection{Results}

\begin{frame}
    \frametitle{Results}
    \framesubtitle{Performance Across Datasets}
    \begin{figure}
        \centering
        \includegraphics[width=0.6\textwidth]{figures/DSTGCRN_tab_results.png}
        \caption{Comparison of DSTGCRN with baselines after 5 runs on datasets from China, the US, and the EU.}
    \end{figure}
\end{frame}



\begin{frame}
    \frametitle{Results}
    \framesubtitle{Long Term Predictions}

    \begin{figure}
        \centering
        \includegraphics[width=\textwidth]{figures/longer_horizon.png}
        \caption{Performance comparison of AGCRN and DSTGCRN across geographies over longer horizons.}
    \end{figure}

\end{frame}

\begin{frame}
    \frametitle{Results}
    \framesubtitle{Factor Analysis}

    \begin{columns}[T] % Top-aligned columns

        % Left column for the first figure
        \begin{column}{0.48\textwidth}
            \begin{figure}
                \centering
                \includegraphics[width=\textwidth]{figures/combined_trend_plot.png}
                \caption{Carbon Emissions Trends from 2019 to 2022 in China.}
            \end{figure}
        \end{column}

        % Right column for the second figure
        \begin{column}{0.48\textwidth}
            \begin{figure}
                \centering
                \includegraphics[width=\textwidth]{figures/comparison_feature.png}
                \caption{Performance metrics across three scenarios.}
            \end{figure}
        \end{column}

    \end{columns}
\end{frame}

\begin{frame}
    \frametitle{Results}
    \framesubtitle{Relationship Evolution and Ablation Studies}

    \begin{columns} % Top-aligned columns

        % Left column for the first figure
        \begin{column}{0.48\textwidth}
            \begin{figure}
                \centering
                \includegraphics[width=\textwidth]{figures/adj_heatmap_stack.png}
                \caption{Differential Temporal Adjacency Matrix Evolution.}
            \end{figure}
        \end{column}

        % Right column for the second figure
        \begin{column}{0.48\textwidth}
            \begin{figure}
                \centering
                \includegraphics[width=\textwidth]{figures/dstgcrn_ablation.png}
                \caption{Ablation experiments on DSTGCRN.}
            \end{figure}
        \end{column}

    \end{columns}
\end{frame}

\section{Probabilistic Traffic Flow Forecasting}

\subsection{Introduction to ProGen}
\begin{frame}
    \frametitle{Introduction}
    \framesubtitle{Background and Challenges}

    \begin{columns}[T]
        \begin{column}{0.48\textwidth}
            \textbf{Background:}
            \begin{itemize}
                \item Spatial-temporal data exhibits complex spatial dependencies and dynamic temporal evolution.
                \item Conventional forecasting models often provide deterministic outputs, which do not account for data uncertainties.
            \end{itemize}
        \end{column}
        \begin{column}{0.48\textwidth}
            \textbf{Challenges:}
            \begin{itemize}
                \item Modeling spatial-temporal interactions is complex due to their intricate structures.
                \item Traditional deterministic approaches fail to handle the probabilistic nature of real-world data, impairing decision-making.
            \end{itemize}
        \end{column}
    \end{columns}
\end{frame}


\begin{frame}
    \frametitle{Introduction}
    \framesubtitle{Motivation and Contributions}

    \begin{columns}[T]
        \begin{column}{0.48\textwidth}
            \textbf{Motivation:}
            \begin{itemize}
                \item Advances in generative AI, especially diffusion models, offer new avenues for probabilistic forecasting that embrace data uncertainty.
                \item ProGen leverages stochastic differential equations to model data’s continuous-time evolution, improving forecast precision.
            \end{itemize}
        \end{column}
        \begin{column}{0.48\textwidth}
            \textbf{Contributions:}
            \begin{itemize}
                \item \textbf{Conceptual:} Pioneering a generative modeling framework for continuous-time spatial-temporal forecasting.
                \item \textbf{Technical:} Develops a unique denoising approach and custom SDE for improved spatiotemporal correlation handling.
                \item \textbf{Empirical:} Proven enhancements over traditional models validated through rigorous real-world data testing.
            \end{itemize}
        \end{column}
    \end{columns}
\end{frame}


\begin{frame}
    \frametitle{Introduction}
    \framesubtitle{Practical Implementation and Forecasting Implications}

    ProGen's implementation of DSM and SDEs offers several key advantages for spatio-temporal forecasting:
    \begin{itemize}
        \item \textbf{Improved Forecast Accuracy:} By accounting for uncertainty and enabling the model to explore a range of possible futures.
        \item \textbf{Robustness to Noise:} The use of SDEs helps handle the inherent noise and variability in spatio-temporal data effectively.
        \item \textbf{Flexibility in Model Application:} Suitable for various types of spatio-temporal data beyond just traffic or weather, including economic and biological datasets.
    \end{itemize}
    Additionally, the continuous-time approach of ProGen allows for finer temporal resolution in predictions, crucial for dynamic systems monitoring and decision-making.
\end{frame}

\subsection{Literature Review}
\begin{frame}
    \frametitle{Literature Review}
    \framesubtitle{Diffusion Models and Probabilistic Time Series Forecasting}

    \begin{columns}[T]
        \begin{column}{0.48\textwidth}
            \textbf{Diffusion Models:}
            \begin{itemize}
                \item Originated by Ho et al. (2020) and Song et al. (2021), using discrete and continuous SDEs to generate data from noise.
                \item Enhanced for conditional generation by Nichol and Dhariwal (2021), aligning outputs with specific attributes.
            \end{itemize}
        \end{column}
        \begin{column}{0.48\textwidth}
            \textbf{Probabilistic Time Series Forecasting:}
            \begin{itemize}
                \item Diffusion models adapted for time series, exemplified by TimeGrad and ScoreGrad, though challenges persist with speed and precision.
                \item ProGen employs a continuous approach, enhancing the capture of spatial and temporal correlations.
            \end{itemize}
        \end{column}
    \end{columns}
\end{frame}

\begin{frame}
    \frametitle{Literature Review}
    \framesubtitle{Spatio-Temporal Forecasting Methods}

    \begin{columns}[T]
        \begin{column}{0.48\textwidth}
            \textbf{Spatio-Temporal Forecasting:}
            \begin{itemize}
                \item Techniques such as AGCRN and DSTAGNN use dynamic graphs; STG-NRDE employs neural rough differential equations.
            \end{itemize}
        \end{column}
        \begin{column}{0.48\textwidth}
            \textbf{Continuity in Time Series:}
            \begin{itemize}
                \item Discrete diffusion models explored probabilistic forecasting but struggled with continuity; ProGen introduces a novel continuous-time method, offering a distinct solution.
            \end{itemize}
        \end{column}
    \end{columns}
\end{frame}


\subsection{Methodology}
\begin{frame}
    \frametitle{Methodology}
    \framesubtitle{Problem Setup}

    Our objective is to forecast future values of a spatio-temporal series based on historical data:
    \begin{itemize}
        \item Define \(\mathcal{D} = \{\mathbf{X_t}\}_{t=1}^T\) where \(\mathbf{X_t} \in \mathbb{R}^{N \times D}\) denotes observations at time \(t\) across \(N\) locations, each with \(D\) features.
        \item Spatial dependencies are modeled through a graph \(\mathcal{G} = (\mathcal{V}, \mathcal{E}, A)\), comprising nodes \(\mathcal{V}\), edges \(\mathcal{E}\), and an adjacency matrix \(A\).
    \end{itemize}
    \vspace{-0.65em}
    \begin{block}{Probabilistic Prediction Task}
        Aim to predict the distribution:
        \[
            q_X (\mathbf{X_{T+1:T+H}} \mid \mathbf{X_{T-L+1:T}}, \mathcal{G}, \mathcal{C})
        \]
        where \(L\) and \(H\) denote the length of the historical window and the forecasting horizon, respectively.
    \end{block}
\end{frame}


\begin{frame}
    \frametitle{Methodology}
    \framesubtitle{Stochastic Differential Equations}

    SDEs provide a framework for modeling continuous-time stochastic processes:
    \begin{equation}
        d\mathbf{X} = f(\mathbf{X}, t)dt + g(\mathbf{X}, t)dW,
    \end{equation}
    where:
    \begin{itemize}
        \item \(\mathbf{X} \in \mathbb{R}^d\) represents the state at time \(t\).
        \item \(f(\mathbf{X}, t)\) is the drift function, dictating deterministic dynamics.
        \item \(g(\mathbf{X}, t)\) is the diffusion function, modeling stochastic effects.
        \item \(dW\) denotes differential Brownian motion.
    \end{itemize}
\end{frame}

\begin{frame}
    \frametitle{Methodology}
    \framesubtitle{Reverse Stochastic Differential Equations}

    Reverse SDEs describe how to denoise data back to its original distribution:
    \begin{equation}
        d\mathbf{X} = \left[f(\mathbf{X}, t) - g^2(\mathbf{X}, t)\nabla_X \log p_t(\mathbf{X})\right]dt + g(\mathbf{X}, t)d\bar{W},
    \end{equation}
    \begin{itemize}
        \item The score function \(\nabla_X \log p_t(\mathbf{X})\) guides the denoising process.
        \item \(\bar{W}\) is the reverse Wiener process, introducing reverse dynamics.
    \end{itemize}
\end{frame}

\begin{frame}
    \frametitle{Methodology}
    \framesubtitle{Denoising Score Matching (DSM)}

    DSM optimizes the match between the gradients of the log probabilities (scores) of the model and data distributions through the diffusion process:
    \begin{equation}
        \mathcal{L}(\theta) = \mathbb{E}_{t \sim \text{Uniform}(0, K)} \mathbb{E}_{X \sim p_{\text{data}}} [\|\nabla_X \log q_{\theta}(\mathbf{X^t} | t) - \nabla_X \log p_{\text{data}}(\mathbf{X^t} | t)\|^2]
    \end{equation}
    where:
    \begin{itemize}
        \item \( q_{\theta}(\mathbf{X^t} | t) \) and \( p_{\text{data}}(\mathbf{X^t} | t) \) are the model and data distributions at diffusion timestep \(t\), respectively.
        \item \( \nabla_X \log \) represents the gradient of the log probability.
        \item \( \theta \) denotes the model parameters optimized during training.
    \end{itemize}
\end{frame}



\begin{frame}
    \frametitle{Overview of ProGen Framework}
    \framesubtitle{Operational Processes}

    ProGen combines a forward diffusion process with a reverse prediction process:
    \begin{itemize}
        \item \textbf{Forward Diffusion:} Transforms training data into a Gaussian state while training a score model.
        \item \textbf{Reverse Prediction:} Iteratively denoises to generate predictions, guided by the score model.
    \end{itemize}

    \begin{figure}[ht]
        \centering
        \includegraphics[width=0.7\textwidth]{figures/ProGen_framework_new.pdf}
        \caption{Overview of the two primary processes in ProGen.}
        \label{fig:framework}
    \end{figure}

\end{frame}

\begin{frame}
    \frametitle{Forward Diffusion Process}
    \framesubtitle{Transforming Data into Gaussian State}

    The forward process perturbs the data point by point into Gaussian noise:
    \begin{equation}
        \mathbf{\tilde{X}^t_{F}} = \mu(\mathbf{X_{F}}, t) + \sigma(\mathbf{X_{F}}, t) \times Z, \quad Z \sim \mathcal{N}(0, I)
    \end{equation}
    where \(\mu\) and \(\sigma\) control the mean and standard deviation across discretized time steps.

    This process trains the model to understand and simulate the transition from real data distributions to noise.
\end{frame}

\begin{frame}
    \frametitle{Training the Denoising Score Model}
    \framesubtitle{Optimizing the Score Estimation}

    Training focuses on minimizing the discrepancy between the estimated and true data gradients:
    \begin{equation}
        \mathcal{L}(\theta) = \mathbb{E}_{t} \left\{ \mathbb{E}_{\mathbf{X}_{F}, \mathbf{X_H}} \left[ \|\nabla \log p(\tilde{\mathbf{X}}_{F}^t | \mathbf{X}_{F}) - s_\theta(\tilde{\mathbf{X}}_{F}^t, \mathbf{X_H})\|^2 \right] \right\}
    \end{equation}
    This loss function aligns the model's score estimates with the true distribution changes, enhancing prediction accuracy.

    \begin{figure}[ht]
        \centering
        \includegraphics[width=0.75\textwidth]{figures/ProGen_model.pdf}
        \caption{Architecture of the Denoising Score Matching Model in ProGen.}
        \label{fig:model}
    \end{figure}

\end{frame}

\begin{frame}
    \frametitle{Adaptive Reverse Prediction Process}
    \framesubtitle{Restoring Original Data Distribution}

    The reverse SDE effectively restores data from its noisy state using learned scores:
    \begin{equation}
        d\mathbf{X} = (f(\mathbf{X}, t) - g(\mathbf{X}, t)^2 \nabla \log p(\mathbf{X}|t))dt + g(\mathbf{X}, t)d\bar{W}
    \end{equation}
    This equation integrates insights from the score model to reverse the diffusion, guiding the data back to its original state.

    ProGen's approach modifies traditional methods by adapting to data dynamics more effectively and efficiently.
\end{frame}

\subsection{Results}
\begin{frame}
    \frametitle{Overview of ProGen Performance}
    \framesubtitle{Full Test Run}

    \begin{figure}[ht]
        \centering
        \includegraphics[width=\textwidth]{figures/det_full_tab.png}
        \caption{Distribution and mean values of predictions vs. actual truths across initial batch iterations in PEMS04.}
    \end{figure}

    \begin{figure}[htbp]
        \centering
        \includegraphics[width=0.35\textwidth]{figures/prob_full_tab.png}
        \caption{Training loss curves for different spatiotemporal layers in PEMS08 over 100 epochs.}
    \end{figure}
\end{frame}

\begin{frame}
    \frametitle{Overview of ProGen Performance}
    \framesubtitle{Random Test Runs}

    \begin{figure}[ht]
        \centering
        \includegraphics[width=0.8\textwidth]{figures/random_runs_tab.png}
        \caption{Distribution and mean values of predictions vs. actual truths across initial batch iterations in PEMS04.}
    \end{figure}
\end{frame}


\begin{frame}
    \frametitle{Overview of ProGen Performance}
    \framesubtitle{Random Test Runs}

    \begin{figure}[ht]
        \centering
        \includegraphics[width=0.8\textwidth]{figures/pems04_3d_waves_plot.pdf}
        \caption{Distribution and mean values of predictions vs. actual truths across initial batch iterations in PEMS04.}
    \end{figure}
\end{frame}



\begin{frame}
    \frametitle{Overview of ProGen Performance}
    \framesubtitle{Random Test Runs}

    \begin{figure}[ht]
        \centering
        \includegraphics[width=0.8\textwidth]{figures/distribution_change_iteration_pems04.png        }
        \caption{Distribution and mean values of predictions vs. actual truths across initial batch iterations in PEMS04.}
    \end{figure}
\end{frame}


\begin{frame}
    \frametitle{Overview of ProGen Performance}
    \framesubtitle{Random Test Runs}

    \begin{figure}[ht]
        \centering
        \includegraphics[width=0.8\textwidth]{figures/train_loss.png        }
        \caption{Distribution and mean values of predictions vs. actual truths across initial batch iterations in PEMS04.}
    \end{figure}
\end{frame}

\begin{frame}
    \frametitle{Overview of ProGen Performance}
    \framesubtitle{Random Test Runs}

    \begin{figure}[ht]
        \centering
        \includegraphics[width=0.8\textwidth]{figures/mae_stsde_subvpsde.png        }
        \caption{Distribution and mean values of predictions vs. actual truths across initial batch iterations in PEMS04.}
    \end{figure}
\end{frame}

\begin{frame}
    \frametitle{Overview of ProGen Performance}
    \framesubtitle{Random Test Runs}

    \begin{figure}[ht]
        \centering
        \includegraphics[width=0.8\textwidth]{figures/crps_stsde_subvpsde.png        }
        \caption{Distribution and mean values of predictions vs. actual truths across initial batch iterations in PEMS04.}
    \end{figure}
\end{frame}

\begin{frame}
    \frametitle{Overview of ProGen Performance}
    \framesubtitle{Random Test Runs}

    \begin{figure}[ht]
        \centering
        \includegraphics[width=0.8\textwidth]{figures/mae_rmse_samples.png       }
        \caption{Distribution and mean values of predictions vs. actual truths across initial batch iterations in PEMS04.}
    \end{figure}
\end{frame}

\begin{frame}
    \frametitle{Overview of ProGen Performance}
    \framesubtitle{Random Test Runs}

    \begin{figure}[ht]
        \centering
        \includegraphics[width=0.8\textwidth]{figures/crps_mis_samples.png        }
        \caption{Distribution and mean values of predictions vs. actual truths across initial batch iterations in PEMS04.}
    \end{figure}
\end{frame}

\begin{frame}
    \frametitle{Overview of ProGen Performance}
    \framesubtitle{Random Test Runs}

    \begin{figure}[ht]
        \centering
        \includegraphics[width=0.8\textwidth]{figures/pems07_mae_rmse_alpha.png        }
        \caption{Distribution and mean values of predictions vs. actual truths across initial batch iterations in PEMS04.}
    \end{figure}
\end{frame}

\begin{frame}
    \frametitle{Overview of ProGen Performance}
    \framesubtitle{Random Test Runs}

    \begin{figure}[ht]
        \centering
        \includegraphics[width=0.8\textwidth]{figures/pems07_crps_mis_alpha.png        }
        \caption{Distribution and mean values of predictions vs. actual truths across initial batch iterations in PEMS04.}
    \end{figure}
\end{frame}



% % % % % % % % % % % % % % % % % % % % % % % % % % % % % % % % % % % %
% % % % % % % % % % % % % % % % % % % % % % % % % % % % % % % % % % % %
% % % % % % % % % % % % % % % % % % % % % % % % % % % % % % % % % % % %
% % % % % % % % % % % % % % % % % % % % % % % % % % % % % % % % % % % %
% % % % % % % % % % % % % % % % % % % % % % % % % % % % % % % % % % % %
% % % % % % % % % % % % % % % % % % % % % % % % % % % % % % % % % % % %
% % % % % % % % % % % % % % % % % % % % % % % % % % % % % % % % % % % %
% % % % % % % % % % % % % % % % % % % % % % % % % % % % % % % % % % % %

\appendix % to start a separate page numbering

\section*{Bibliography}
\begin{frame}[allowframebreaks]
    \textbf{References}
    \printbibliography
\end{frame}


{ % all template changes are local to this group.
\setbeamertemplate{navigation symbols}{}
\begin{frame}<article:0>[plain,noframenumbering]
    \begin{tikzpicture}[remember picture,overlay]
        \node[at=(current page.center)] {
            \includegraphics[
                width=\paperwidth,
                height=\paperheight]{figures/ThankYouPage.png}
        };
    \end{tikzpicture}
    % \begin{tikzpicture}[remember picture,overlay]
    %     \node[at=(current page.center)] {
    %         \includegraphics[keepaspectratio,
    %             width=0.65\paperwidth,
    %             height=\paperheight]{figures/ThankYouPage.png}
    %     };
    % \end{tikzpicture}
\end{frame}
}

\end{document}
